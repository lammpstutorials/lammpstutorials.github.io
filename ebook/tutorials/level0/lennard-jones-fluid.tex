\chapter{Lennard Jones fluid}

\vspace{-1cm} \noindent \textcolor{graytitle}{\textit{{\Large The very basics of LAMMPS through a simple example}}\vspace{0.5cm} }

The objective of this tutorial is to use
LAMMPS to perform a simple molecular dynamics simulation
of a binary fluid in the NVT ensemble. The system is a simple Lennard-Jones fluid
made of neutral dots with a Langevin thermostating. The
simulation box is cubic with periodic boundary conditions.
This tutorial illustrates the use of several ingredients of
molecular dynamics simulations, such as system initialization,
energy minimization, integration of the equations of motion,
and trajectory visualization.

Download and install LAMMPS by following the instructions of the |LAMMPS website|.
Alternatively, if you are using Ubuntu OS, you can simply execute the
following command in a terminal:

You can verify that LAMMPS is indeed installed on your
computer by typing in a terminal :

You should see the version of LAMMPS that has been
installed. On my computer I see

In addition to LAMMPS, you will also need |(1) a basic text editing software|
such as Vim, Gedit, or Notepad++, |(2) a visualization software|, here I
will use VMD (note: VMD is free but you have to register to
the uiuc website in order to download it. If you don't want
to, you can also use Ovito.), |(3) a plotting tool| like
XmGrace or pyplot.

In order to run a simulation using LAMMPS, one needs to
write a series of commands in an input script. For clarity,
this script will be divided into five categories which we are going to
fill up one by one. Create a blank text file, call it
*input1.lammps*, and copy the following lines in it:

These five categories are not required in every
input script, and should not necessarily be in that
exact order. For instance parts 3 and 4 could be inverted, or
part 4 could be omitted, or there could be several
consecutive runs.
A line starting with a brace (#) is a comment
that is ignored by LAMMPS. Use comments to structure 
your inputs and make them readable by others.

In the first section of the script, called *Initialization*,
let us indicate to LAMMPS the type of simulation we are
going to execute by specifying the most basic information,
such as the conditions at the boundaries of the box (i.e.
periodic, non-periodic) or the type of atoms (e.g. uncharged
single dots, spheres with angular velocities). Enter the
following lines:

The first line indicates that we want to
use the system of unit called *lj*, for lennard-jones, for which all quantities
are unitless. 


The second line indicates that the simulation
is 3D, the third line that the *atomic* style
will be used, therefore each atom is just a dot with a mass.


The fourth line indicates that atoms are going to interact
through a Lennard-Jones potential with a cut-off equal to
2.5 (unitless), and the last line indicates that the
periodic boundary conditions will be used along all three
directions of space (the 3 *p* stand for *x*, *y*, and *z*,
respectively).
At this point, you have a LAMMPS script that does nothing.
You can execute it to verify that there is no mistake by
running the following command in the terminal:

Which should return something like

If there is a mistake in the input script, for example if
*atom_stile* is written instead of *atom_style*, LAMMPS
gives you an explicit warning:

Let us fill the second part the of the input script:

The first line creates a region of space
named *simulation_box* that is a block (a rectangular cuboid) and
extends from -20 to 20 along all 3 directions of space, all expressed in
non-dimensional units because we are using the lj system
of units. The second line creates a simulation box based on
the region *simulation_box* with *2* types of atoms. The third
command specifies that 1500 atoms of type 1 must be created
randomly in the region *simulation_box*. The integer *341341* is a
seed that can be changed in order to create different
initial conditions for the simulation. The fourth line
creates 100 atoms of type 2.
If you run LAMMPS, you should see the following in the
terminal:

From what is printed in the terminal, it is clear that
LAMMPS correctly interpreted the commands, and first created
the box with desired dimensions, then 1500 atoms, then 100
atoms.
Let us fill the third section of the input script, the settings:

The two first commands attribute a mass
equal to 1 (unitless) to both atoms of type 1 and 2,
respectively. The third line sets the Lennard-Jones
coefficients for the interactions between atoms of type 1,
respectively the depth of the potential well
:math:`\epsilon` and the distance at which the
particle-particle potential energy is zero :math:`\sigma`. 
The last line sets the Lennard-Jones coefficients for
the interactions between atoms of type 2.


Now that the system is fully defined, let us just fill the two last remaining sections:

The thermo command asks LAMMPS to print
thermodynamic information (e.g. temperature, energy) in the
terminal every 10 steps. The second line asks LAMMPS to
perform an energy minimization of the system.


Now running the simulation, we can see how the thermodynamics
variables evolve with time:

These lines give us information concerning
the progress of the energy minimization. First, at the start
of the simulation (step 0), the energy in the system is
huge: 78840982 (unitless). This was expected because
the atoms have been created at random positions within the
simulation box, and some of them are probably overlapping,
resulting in a large initial energy which is the consequence
of the repulsive part of the Lennard-Jones interaction
potential. As the energy minimization progresses, the energy
rapidly decreases and reaches a negative value, indicating that the atoms have been
displaced at reasonable distances from one another. Other
useful information have been printed in the terminal, for
example, LAMMPS tells us that the first of the four criteria
to be satisfied was the energy:


The system is now ready. Let us continue filling up the
input script and adding commands in order to perform an actual molecular dynamics
simulation that will start from the final state of the energy minimization.
In the same input script, after the *minimization* command, add the following
lines:

Since LAMMPS reads the input from top to
bottom, these lines will be executed after the energy
minimization. There is no need to (re-)initialize the system
(part 1), (re-)define it (part 2), or (re-)specify the settings
(part 3). The *thermo* command is called a second time within the 
same input, so the previously entered value of *10* will be replaced
by the value of *1000* as soon as the second run starts.

Three variables have been defined in order
to print the kinetic energy and the potential energy 
of the system in the file named *energy.dat*. Then,
in the run section, the fix *nve* is used to update the
positions and the velocities of the atoms in the group
*all* (this is the most important command here). The second
fix applies a Langevin thermostat to the atoms of group
*all*, with a desired temperature of 1 and a damping
parameter of 0.1. The number *1530917* is a seed, you can
change it to perform statistically independent simulations
with the same system. Finally we choose the timestep
and we ask LAMMPS to run for 10000 steps. After running
the simulation, you should see the following information in
the terminal:

The second column shows that the temperature
starts from 0, but rapidly reaches the
expected value near :math:`T=1`, as requested. 
Note that  In the terminal, you may also see

Note : If you see *Dangerous builds = 0*, as could be
the case with some LAMMPS versions, you can ignore
the next part.
During the simulation, they have been 998 dangerous builds.
This is an indication that something is wrong: some atoms
have moved more than expected in between two calculations of
the neighbor lists. Let us add the following command in the
*Simulation settings* section:

With this command, LAMMPS will rebuild the neighbor lists
more often. Re-run the simulation, and you should see a more
positive outcome with 0 dangerous build:

From what has been printed in the energy.dat file, let us
plot the potential energy and the pressure of
the system over time:


The simulation is running well, but we would like to
visualize the trajectories of the atoms. To do so, we need
to dump the positions of the atoms in a file at a regular
interval. Add the following command in the *visualization*
section of PART B:

Run LAMMPS again. A file named *dump.lammpstrj* must appear in
the same folder as your input. This file can be opened using
VMD or Ovito. In Ubuntu, if VMD is installed, you can simply
execute in the terminal:

Otherwise, you can open VMD and import the *dump.lammpstrj*
file manually using file *->* molecule. You should see a cloud
of lines, but you can improve the representation and make it
look like the figure on the right, or the video at the 
top of this page. 
Improving the script
====================

Let us improve the input script and perform slightly more
advanced operations.

Let us create the atoms of type 1 and 2 in two separate
regions, respectively, instead of creating them both randomly 
within the entire space as we did previously. Create a new input script, and call
it *input2.lammps*. Similarly to what has been done previously, copy the following lines
into the input script:

Let us create a box from a predefined region,
and create two additional regions and generate
atoms of type 1 and 2 in each region respectively.

The *side in* and *side out* keywords
allow us to define regions that are respectively inside the
cylinder, and everything that is not inside the cylinder.
We can write the remaining of the input script as follow:

The novelty with respect to the previous
input script is the command *write_data*. This command
asks LAMMPS to print the final state of the simulation in
a file named *minimized_coordinate.data*. This file will
be used later to restart the simulation from the final
state of the energy minimisation step.
Run LAMMPS using the *input2.lammps* script. If everything
goes well, a dump file named *dump.min.lammpstrj* will
appear in the folder, allowing you to visualize the atoms
trajectories during minimization. In
addition, a file named *minimized_coordinate.data* will be
created. If you open this file, you will see that it
contains all the information necessary to restart the
simulation, such as the number of atoms and the size of
the box:

The *minimized_coordinate.data* file also contains the final
positions and velocities of all the atoms:

The columns of the Atoms section
correspond (from left to right) to the atom indexes (from 1
to the total number of atoms, 1150), the atom types (1 or 2
here), the atoms positions :math:`x`, :math:`y`, :math:`z` and the
atoms velocities :math:`v_x`, :math:`v_y`, :math:`v_z`.
Restarting from a saved configuration
-------------------------------------

We are going to create a new input file and start a
molecular dynamics simulation directly from the previously
saved configuration. In the same folder, create a new file
named input3.lammps and copy the same lines as previously:

Now, instead of creating a new region and adding atoms, we
simply add the following command:

By visualizing the previously generated dump.min.lammpstrj
file, you may have noticed that some atoms have moved from
one region to the other during minimisation, as seen in
|this video|.
In order to start the simulation from a clean state, with
only atoms of type 2 within the cylinder and atoms of type
1 outside the cylinder, let us delete the misplaced atoms
by adding the following commands:

These commands will respectively recreate
the previously defined regions (regions are not saved by the
*write_data* command), create groups, and finally delete the
atoms of type 1 that are located within the cylinder, as
well as the atoms of type 2 that are located outside the
cylinder. If you run LAMMPS, you can see in the terminal how
many atoms are in each group, and how many atoms have been
deleted:

Similarly to previously, add the following simulation
settings:

Note that 2 atom groups have been defined, they are useful
here to extract the coordination number between atoms of
type 1 and 2. Let us extract this coordination number, as
well as the number of atoms of each type in each region, by
adding the following commands to the input file:

As seen previously, the fix ave/time
allow to evaluate previously defined variables and print
the values (here every 2000 steps, after averaging each quantities 200 times)
into data file. The 4 variables starting with *Ntype* are used to count
the number of atoms of a specific group in a specific
region. 
Let us also extract the coordination number per atom between atoms 
of type 1 and 2, i.e. the average number of atoms of type 2 in the vicinity 
of the atoms of type 1. This coordination number will be used as an indicator of the 
degree of mixing of our binary mixture. 

The *compute ave* is used to average the per atom
coordination number that is calculated by the *coord/atom* compute.
This averaging is necessary as *coord/atom* returns an array where each value corresponds 
to a certain couple of atom i-j. Such array can't be printed by *fix ave/time*. 
Finally, let us complete the script by adding the run section:

There are a few differences with the
previous input script. First, the *velocity create*
command attributes an initial velocity to all the atoms.
The initial velocity is chosen so that the initial
temperature is equal to 1 (unitless). The additional
keywords ensure that no linear momentum and no angular
momentum are given to the system, and that the generated
velocities are distributed as a Gaussian. Another novelty
is the *zero yes* keyword in the Langevin thermostat, that
ensures that the total random force is equal to zero.
After running the simulation, you can observe the number
of atoms in each region from the generated data files, as
well as the evolution of the coordination number due to
mixing:

Now that you have completed this simple molecular dynamics tutorials, what can you do?
Play around
-----------

A good way to progress with LAMMPS and molecular dynamics
simulations is to play around with a script that is already
working and observe the differences and/or errors occurring:
Try adding new commands (you can choose from the documentation),
try removing some of the commands, try changing the parameter values
(see also the first exercise below).
The more you trigger warnings, the easier it will be for you to solve your
own simulation.
Try another tutorial
--------------------

There are many common aspects of molecular simulations that were not dealt with in this
tutorial:
- dealing with charged atoms and bounded molecules, as is necessary to model most existing molecules, solids, or structures, see for instance :ref:`all-atoms-label` and :ref:`sheared-confined-label`,
- dealing with non-constant volume,
- dealing with reactivity and bond formation/breaking, see :ref:`reactive-silicon-dioxide-label`.
Exercises
=========

So far, simulations were made using the NVT ensemble [constant number 
of atoms, N, constant volume V, and constant (or at least imposed)
temperature T].
Run a similar simulation in the NVE ensemble, and extract the
total energy of the system over time.


Run a successful simulation without using the *minimize* command.
The absence of energy minimization needs to be compensated
in order to avoid triggering the *ERROR: Lost atoms* message.


So far, atoms were freely diffusing without contraint or external force.
Add an external force to induce a net flow of atoms in one
direction. The magnitude of the force must be chosen so
that the system is not *too far* from equilibrium.


Add a bond between couple of identical atoms to create
dumbbell molecules, just like in the image:


